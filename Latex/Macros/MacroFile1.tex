% This file contains macros that can be called up from connected TeX files
% It helps to summarise repeated code, e.g. figure insertion (see below).

% insert a centered figure with caption and description
% parameters 1:filename, 2:title, 3:description and label
\newcommand{\figuremacro}[3]{
	\begin{figure}[htbp]
		\centering
		\includegraphics[width=1\textwidth]{#1}
		\caption[#2]{\textbf{#2} #3}
		\label{#1}
	\end{figure}
}

% insert a centered figure with caption and description AND WIDTH
% parameters 1:filename, 2:title, 3:description and label, 4: textwidth
% textwidth 1 means as text, 0.5 means half the width of the text
\newcommand{\figuremacroW}[4]{
	\begin{figure}[htbp]
		\centering
		\includegraphics[width=#4\textwidth]{#1}
		\caption[#2]{\textbf{#2} #3}
		\label{#1}
	\end{figure}
}
% --------------jess------- Jacob.modificacion ----------Inicia-----------------------

%jacob.modif  jacob.modificado jacob.agregado
% insert a centered figure with caption and description AND WIDTH
% parameters 1:filename, 2:title, 3:description and label, 4: textwidth
% textwidth 1 means as text, 0.5 means half the width of the text
\newcommand{\figuremacroJ}[5]{
	\begin{figure}[htbp]
		\centering
		\includegraphics[width=#4\textwidth]{#1}
		\caption[#2]{\textbf{#2} #3}
		\label{#5}
	\end{figure}
}

\newcommand{\figmacroJ}[5]{
	\begin{figure}[htbp]
		\centering
		\includegraphics[scale=#4]{#1}
		\caption[#2]{\textbf{#2} #3}
		\label{#5}
	\end{figure}
}


\newcommand{\figmacroubicJ}[6]{
	\begin{figure}[#6]
		\centering
		\includegraphics[scale=#4]{#1}
		\caption[#2]{\textbf{#2} #3}
		\label{#5}
	\end{figure}
}


%macro para incluir una imagen en posición específica (no tipo float)

\newcommand{\graphicsmacroJ}[5]{
	\noindent
	\ \\
	\begin{minipage}{\textwidth}
		\centering
		\includegraphics[scale=#4]{#1}
		\captionof{figure}[#2]{\textbf{#2} #3} \label{#5}
	\end{minipage}
	
	\ \\
}

%macro para incluir una imagen en posición específica con un texto junto a ella
%el texto es la línea 1
\newcommand{\TgraphicsmacroJ}[6]{

	\noindent
	\begin{minipage}{\textwidth}
		#1  \\ \par
		\centering
		\includegraphics[scale=#5]{#2}
	\captionof{figure}[#3]{\textbf{#3} #4} \label{#6}
	\end{minipage}
	\ \\
}


% Arrays
\newcommand{\jarp}[2]{\left( \begin{array}{#1} #2 \end{array}\right)}
\newcommand{\jarc}[2]{\left[ \begin{array}{#1} #2 \end{array}\right]}

%macro para incluir una derivada parcial
\newcommand{\jpar}[2]{
	\frac{\partial{#1}}{\partial{#2}}
}
%bold symbols
\newcommand{\bsym}[1]{
	\boldsymbol{#1}
}


\newcommand{\argmin}{\arg\!\min}



% --------------jess------- Jacob.modificacion ----------Termina-----------------------
% --------------jess------- Jacob.modificacion ----------Termina-----------------------
% --------------jess------- Jacob.modificacion ----------Termina-----------------------
% --------------jess------- Jacob.modificacion ----------Termina-----------------------


% inserts a figure with wrapped around text; only suitable for NARROW figs
% o is for outside on a double paged document; others: l, r, i(inside)
% text and figure will each be half of the document width
% note: long captions often crash with adjacent content; take care
% in general: above 2 macro produce more reliable layout
\newcommand{\figuremacroN}[3]{
	\begin{wrapfigure}{o}{0.5\textwidth}
		\centering
		\includegraphics[width=0.48\textwidth]{#1}
		\caption[#2]{{\small\textbf{#2} - #3}}
		\label{#1}
	\end{wrapfigure}
}

% predefined commands by Harish
\newcommand{\PdfPsText}[2]{
  \ifpdf
     #1
  \else
     #2
  \fi
}

\newcommand{\IncludeGraphicsH}[3]{
  \PdfPsText{\includegraphics[height=#2]{#1}}{\includegraphics[bb = #3, height=#2]{#1}}
}

\newcommand{\IncludeGraphicsW}[3]{
  \PdfPsText{\includegraphics[width=#2]{#1}}{\includegraphics[bb = #3, width=#2]{#1}}
}

\newcommand{\InsertFig}[3]{
  \begin{figure}[!htbp]
    \begin{center}
      \leavevmode
      #1
      \caption{#2}
      \label{#3}
    \end{center}
  \end{figure}
}


%---Macros Muttio---%

%El siguiente macro sirve para darle un cuadro tipo teorema a algun texto
%utilizado en ejercicios resueltos.
%Solo debe ponerse el texto dentro del parentesis despues de \cuadtext{}

\newcommand{\cuadtext}[2]{
    \begin{figure}[htbp]
    \begin{theo #1}
    #2
    \end{theo #1}
    \end{figure}
}

%El siguiente macro es utilizado en conjunto con el anterior
%sirve para agregar imagenes dentro de cuadros de texto
%notar que es el mismo que \figmacroJ solo que sin el ambiente
%figure que tiene conflicto al usar el ambiente theo

\newcommand{\figmacroCT}[5]{	
		\centering
		\includegraphics[scale=#4]{#1}
		\caption[#2]{\textbf{#2} #3}
		\label{#5}
}




%%% Local Variables: 
%%% mode: latex
%%% TeX-master: "~/Documents/LaTeX/CUEDThesisPSnPDF/thesis"
%%% End: 
